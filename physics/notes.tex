% Created 2025-07-27 Sun 13:02
% Intended LaTeX compiler: pdflatex
\documentclass[11pt]{article}
\usepackage[utf8]{inputenc}
\usepackage[T1]{fontenc}
\usepackage{graphicx}
\usepackage{longtable}
\usepackage{wrapfig}
\usepackage{rotating}
\usepackage[normalem]{ulem}
\usepackage{amsmath}
\usepackage{amssymb}
\usepackage{capt-of}
\usepackage{hyperref}
\author{Mr. Maxwell}
\date{\today}
\title{Notes}
\hypersetup{
 pdfauthor={Mr. Maxwell},
 pdftitle={Notes},
 pdfkeywords={},
 pdfsubject={},
 pdfcreator={Emacs 30.1 (Org mode 9.7.11)}, 
 pdflang={English}}
\begin{document}

\maketitle
\tableofcontents

"Science and engineering are based on measurements and comparisons." (Jearl Walker, David Halliday, Robert Resnick, 2014 p. 1)
\section{Measurement                                                      :unit 1:}
\label{sec:org2fd51d7}

\subsection{SI units}
\label{sec:org524d4bf}

\begin{equation}
\label{eq:org1f99bc6}
    1 \; watt = 1\; W = 1 \; kg \cdot m^2/s^3
\end{equation}
\subsubsection{length}
\label{sec:org7557978}

The meter is defined as the distance traveled by light during a precisely specified time interval.
\subsubsection{time}
\label{sec:org530a48e}

\subsubsection{mass}
\label{sec:org258b034}

\subsubsection{test}
\label{sec:org4bdbaf5}

\ref{eq:org1f99bc6}
\end{document}
